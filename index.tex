\documentclass{article}

\usepackage{environ}
\usepackage{makeidx}
\usepackage{hyperref}
\usepackage{minted}
\usepackage[dvipsnames]{xcolor}

\usepackage{totcount}
\regtotcounter{section}
\makeatletter
	\renewcommand{\thesection}{\number\numexpr\c@section@totc-\c@section+1\relax}
\makeatother
\providecommand{\sectionautorefname}[1]{Post ~}

\providecommand{\post}[2]{\HCode{<hr>}\Link{#2}{#2}\section{#1}\EndLink}

\definecolor{commentgray}{gray}{0.5}
\NewEnviron{comment}{
	\begin{quote}
	\textcolor{commentgray}{\BODY}
	\end{quote}}

\setlength{\parindent}{0ex}


\title{Chiel Douwes' TMP blog}
\date{}

\begin{document}

\maketitle


\post{Introduction}{intro}
Just in time for the christmas festivities, I bring you: my TMP blog!
My name is Chiel Douwes and you might know me as the maintainer of the current fastest
metaprogramming library \href{https://github.com/kvasir-io/mpl}{kvasir::mpl}, author of the first
parameterized metaprogramming unit testing library \href{https://github.com/chieltbest/metacheck}{metacheck},
or just that one guy. Anyhow, as you are currently reading
this blog, you probably have some intrest in it which is why I'd like to give a short
introduction on what this blog is going to contain and how it is produced.

\subsection*{Content}
All of the posts here will be concise (as you can probably tell by this post) and informative,
about various subjects that interest me, which will probably mostly be all things meta.
\begin{comment}
In fact, this first post is in essence a meta-post
\end{comment}
When I'm working on a project and it could interest other people, I will probably make a post
about it here for your enjoyment.
\begin{comment}
Of which there already are several subjects lined up
\end{comment}

\subsection*{Production}
This website is made using \href{https://github.com/michal-h21/make4ht}{make4ht}
which can generate html files from \LaTeX sources. Make4ht (with tex4ht as backend) supports all the
commands needed for the packages used, and has extra commands to add custom html to the code.
There are some packages that are used to achieve special requirements, like \verb|minted| for code
highlighting:
\begin{minted}{cpp}
template<template<typename...> class F, typename State, typename ...>
struct fold_impl {
    using type = State;
}

template<template<typename...> class F, typename State, typename T, typename ...Ts>
struct fold_impl {
    using type = fold_impl<F, F<State, T>, Ts...>;
}
\end{minted}
Packages like \verb|environ| and \verb|xcolor| are also used to make fancy block comments.
\begin{comment}
Like this one
\end{comment}
And because the default font is kind of lame and inconsistent across devices, I opted to use
the \href{http://mozilla.github.io/Fira/}{Fira Sans} font for text,
and the awesome \href{https://github.com/tonsky/FiraCode}{Fira Code} for monospace text.\\
All this is built by use of a CMake script, keeping track of the different files that have
changed and allowing for easier integration into my editor.

The entire website is available on \href{https://github.com/chieltbest/website}{github}, where
you can find the build script, the configuration and also all the posts in raw source form.

\subsection*{Conclusion}
In conclusion, making a website in latex is a fun, but somewhat impractical way to put stuff on
the internet. Since I like fun and impractical things this seems like the perfect exercise, but
if you're aiming at an easily maintainable website then you should probably go for some other
solution. If you have any comments to make about the layout or readability of this website, feel
free to leave it on \href{https://twitter.com/chieltbest}{twitter}, as I'm not planning to
implement a comment section any time soon.\\
More posts will follow soon(ish) about subjects that are more closely related to c++.

\end{document}

%\abstract{test}
